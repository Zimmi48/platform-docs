\documentclass{easychair}

\usepackage{blindtext}
\usepackage{enumitem}
\usepackage{hyperref}

\title{Coq Platforms docs: A Compilation of Short Interactive Tutorials and
How-To Guides for Coq}

\date{}

\author{
  Thomas Lamiaux\inst{1}
  \and
  Pierre Rousselin\inst{2}
  \and
  Théo Zimmermann\inst{3}
}

\institute{
  ENS Paris-Saclay \quad
  \email{thomas.lamiaux@ens-paris-saclay.fr}
  \and
  LAGA, Université Sorbonne Paris Nord \quad
  \email{rousselin@math.univ-paris13.fr}
  \and
  LTCI, Télécom Paris, Polytechnic Institute of Paris \quad
  \email{theo.zimmermann@telecom-paris.fr}
}

\authorrunning{Lamiaux, Rousselin and Zimmermann}
\titlerunning{Coq Platform docs}

\begin{document}

\maketitle

%\begin{abstract}
%  We introduce Coq Platform docs, a project that aims to build a comprehensive
%  documentation for Coq and its platform through an online compilation of short
%  and interactive tutorials and how-to guides for Coq and the Coq Platform.
%\end{abstract}

%\section{Coq Platform docs}

\section{Motivation}

Having a proper, clean and accessible documentation is one of the keys to the
success of software.
There are different forms of documentation: abstract and detailed documentation
like the reference manual~(\cite{Link_Coq_Ref}), course-shaped documentation like
Coq'Art~(\cite{bertot2013interactive}) or
Software Foundations~(\cite{Pierce:SF1}), or short action-oriented
documentation. Here, we focus on the latter.
Short action-oriented documentation provides users with practical information on
specific features of Coq, such that users can both learn and discover new
features by themselves, but also consult them when they are trying to use
features and are stuck.

Apart from long, course-shaped books, Coq and the packages from its ecosystem currently have only minimal and
scattered action-oriented documentation.
The reference manual is by design not learning-oriented and not action-oriented,
and it would be a mistake to try to bend it that way.
Books like Coq'Art or Software Foundations provide nice pedagogical explanations
but target specific audiences, and often read from cover to cover.
Moreover, they are not well suited to learn about specific features, to discover
horizontally, and are not easy to keep updated.

Yet, short action-oriented documentation has many interests:
\begin{itemize}[itemsep=0pt]
  \item Having an easy to access documentation, accessible through a nice
        centralized online interface is of utmost importance to engage new
        users, and keep current users.
        We cannot expect users to have to dig on their own through the
        reference manual, books, or GitHub repositories of ecosystem packages to learn how to use or
        get information while working about a specific feature.
        Most particularly as these sources may not contain the basic answers
        they are looking for, due to their nature.
  \item Not having such a documentation prevents people from actually
        discovering and learning by themselves new amazing features, as well as
        the richness of our ecosystem~\cite{appel2022coq}.
        Indeed, many features and packages are still currently under-documented, and
        when existing, the documentation is often scattered out, making it hard
        to discover a feature if one is not already an expert.
        A symptom of that is the trouble that students are currently facing to
        find answers or discover new functionalities by themselves, even sometimes
        about basic features.
  \item Writing proper documentation forces us to explain the different aspects
        of a feature clearly, step by step, thus to understand features better.
        Therefore, we hope that by writing the documentation, we will clarify
        the use of many features, and potentially discover or shed light on bugs.
        Actually, writing tutorials for Equations has already revealed different
        issues with the main tactic \href{https://github.com/Zimmi48/platform-docs/pull/1}{funelim}
        and bugs involving \href{https://coq.zulipchat.com/#narrow/stream/237659-Equations-devs-.26-users/topic/Bug.20funelim.20on.20Ack}{rewrite}.
  \item Most users are currently unaware of the extent of what has been
        formalised and is available in Coq.
        There are many libraries, and it is not easy to know which library to
        use, or to know on which axioms they rely or their compatibilities.
        This is obviously not just a documentation issue, but having a clearer
        documentation of what we have and where would help.
\end{itemize}

\section{Description of the project}

This project aims to create an online compilation of short and interactive
tutorials and how-to guides for Coq and the Coq Platform~\cite{Link_Coq_Platform,palmskog2022reliably}.
Each core functionality and plugin of Coq and the Coq Platform should have
(short) pedagogical tutorials and/or how-to guides demonstrating how to use the
functionality, with practical examples. They should further be available online
through an interactive interface, most likely using JsCoq~\cite{jscoq},
and a non-interactive interface (for reading on mobile devices).

Tutorials and how-to guides serve different purposes and are complementary.
Tutorials guide a user during learning in discovering specific aspects of a
feature like ``Notations in Coq'', by going through (simple) predetermined
examples, and introducing notions gradually. In contrast, how-to guides are
use-case-oriented and guide users through real life problems and their inherent
complexity, like ``How to define functions by well-founded recursion and reason
about them''.

\paragraph{Advantages of such a documentation}

Such form of documentation is complementary to other kinds of documentation like
the reference manual, and has several advantages:

\begin{itemize}[itemsep=0pt]
  \item Tutorials should enable users to learn and discover specific features on
        their own, modularly, and according to their needs.
  \item How-tos should provide users practical answers to practical problems
        that they can refer to when working.
  \item By nature, the documentation should be mostly horizontal, which should:
    \begin{itemize}[itemsep=0pt]
      \item make it easy to navigate and to find specific information,
      \item prevent users from having to read a bunch of documentation to be
	    able to read a specific tutorial,
      \item make it possible to build it gradually, making new tutorials and
	    how-tos available as we progress,
      \item allow differentiated learning: depending on your background or
            objective you can navigate the documentation differently,
            potentially reading different tutorials.
    \end{itemize}
  \item It will enable us to showcase all that is possible in Coq's ecosystem.
  \item It should be easy to maintain, as once fully written, a tutorial or how-to
        should not have any reason to change, except if the associated feature or
        known best practices change.
\end{itemize}

\paragraph{Current status}

There now is a \href{https://github.com/Zimmi48/platform-docs}{GitHub repository}
that people can check out to discover the project, and a dedicated
\href{https://coq.zulipchat.com/#narrow/stream/437203-Platform-docs}{Zulip stream}
to discuss the project.

A few tutorials have already been written and are available on the repository.
They have already proven useful by providing practical answers that are
otherwise hard to find without asking directly to an expert.

A first online interactive interface based on JsCoq and coqdoc is currently
being developed and will be available soon.
In the future, we hope to support a more standard and expressive format,
possibly using Alectryon~\cite{pit2020untangling}.

A Coq Enhancement Proposal is currently being written to discuss the
project further with the community.
We hope that, eventually, this documentation will become part of the website that will be created when renaming Coq to \emph{the Rocq Prover}.

\label{sect:bib}
\bibliographystyle{plain}
\bibliography{platformdoc}

\end{document}
